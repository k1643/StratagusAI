\documentclass{article}
\usepackage{amsmath}
\usepackage[pdftex]{graphicx}
\usepackage{url}

\title{StratagusAI Manual}
\date{Sep. 6, 2011}
\author{Brian King}

\begin{document}
\maketitle

\section*{Introduction}
The StratagusAI project contains Java projects that form a modular game player for the Stratagus game engine.  This manual describes 
some of the tools needed to build a Stratagus client in StratagusAI.

\section*{Tools}

\subsection*{Maven Build System}
StratagusAI Java projects are built using the Maven is a build tool.  In Maven, a developer declares the properties of a Java project and its dependencies 
on Java code libraries (JAR files) in a Project Object Model (\verb+pom.xml+) file.  One of the most useful features of Maven is its ability to 
download 3rd party JAR files from an online repository.

Maven is available at \url{http://http://maven.apache.org/download.html}.

\subsection*{GLPK (Gnu Linear Programming Kit)}

The tac-lp tactical combat module assigns combat units to targets by solving a linear program. The solver is GLPK (Gnu Linear Programming Kit),
which is a binary executable that can be called from Java using a GLPK JAR.  The binary and the JAR have to be installed manually.

GLPK can be found at \verb+http://ftp.gnu.org/gnu/glpk/+.  On Windows copy \verb+glpk_4_45.dll+ and \verb+glpk_4_45_java.dll+
to \verb+C:\Windows\System32+. Install the Java JAR using

\begin{verbatim}
mvn install:install-file -DgroupId=org.gnu.glpk \
         -DartifactId=glpk-java -Dversion=4.45 \
         -Dpackaging=jar -Dfile=glpk-java.jar
\end{verbatim}

After the executable library and JAR are installed, you can include GLPK in a Java project with the following entry in the Maven \verb+pom.xml+ file:

\begin{verbatim}
    <dependency>
        <!-- linear program solver -->
        <groupId>org.gnu.glpk</groupId>
        <artifactId>glpk-java</artifactId>
        <version>4.45</version>
    </dependency>
    ...
\end{verbatim}

\section*{High-level Modules}

\section*{Implementation Modules}

\end{document}